\chapter*{Introduction}
\addcontentsline{toc}{chapter}{Introduction}
\markboth{Introduction}{Introduction}
\label{chap:introduction}
%\minitoc

Ce projet de fin d'études s'inscrit dans le cadre d’un projet de recherche lié à l’aide à la décision pour les opérations de recherche et sauvetage sur le territoire canadien. L’objectif du projet est le développement d’algorithmes pour assurer l’utilisation efficace et optimale des ressources de recherche affectées à la localisation d'un aéronef, d'un vaisseau ou de personnes portées disparues lors d'incidents maritimes. Il s'agit de programmer, tester, valider et documenter des algorithmes de partitionnement de données géolocalisées permettant d’identifier en un temps raisonnable de l'ordre de quelques minutes les régions avec la plus haute probabilité de contenir les éléments ou victimes recherchées. Ce projet est susceptible de conduire à une thèse financée. 

\paragraph{}
Ce stage nécessitera d'utiliser les compétences suivantes : 
\begin{itemize}
	\item Connaissance des principes d'apprentissages artificiels de clustering non supervisés de données géolocalisées,
	\item Implémentation en C++,
	\item Connaissances des Systèmes d'informations géographiques (GIS) tels que QGIS
	pour la visualisation des résultats. 
\end{itemize}  

\paragraph{}
Dans cette version 0.1, constituant le 1er livrable, on trouvera une première exploration de la problématique et une étude bibliographique sur les algorithmes non supervisés 