\subsection{Clients}

La nature des clients dépend fortement du type de modèle.

Les intégrateurs s'adressent directement aux entreprises ou aux individus, lesquels attachent une importance particulière à la logistique simplifiée porte-à-porte et aux délais de livraison garantis par une offre multimodale riche. Par ailleurs, les principaux intégrateurs ne se limitent pas au transport aérien et peuvent espérer tirer profit d'autres activités et toucher une clientèle élargie et fidélisée pour laquelle le transport aérien n'est plus une fin.

Les clients directs des compagnies de cargo ou des compagnies mixtes seront des logisticiens qui disposent des infrastructures et équipements nécessaires à la manutention du fret. Le surplus dans l'offre en capacité de fret conduit à une situation où le client
possède un fort pouvoir de négociation : non captif, le client peut changer facilement de fournisseur de service de fret aérien. Ainsi, les compagnies de fret n'ont parfois d'autre choix que de brader leurs prix.

En terme de taille de marché, les États-Unis en 2016 arrivent en tête avec 7.7 millions de tonnes de fret, suivis par l'Allemagne (4.4 Mt), la Chine (3.4 Mt) et Hong-Kong (3.2 Mt), de source IATA.

